\RequirePackage{luatex85}

\documentclass[preprint,prb,amsmath,superscriptaddress,showpacs]{revtex4}

%\usepackage[utf8]{inputenc}
% utf8 is not necessary once we use luatex engine

\usepackage{graphics}
\usepackage{graphicx}% Include figure files
\usepackage{dcolumn} % Align table columns on decimal point
\usepackage{bm}
\usepackage{epsfig}
\usepackage{physics}
\usepackage[linktoc=all]{hyperref}
\usepackage{natbib}
\usepackage{bibentry}
\usepackage{ bbold }
\usepackage{pgf,tikz}
\usepackage[compat=1.1.0]{tikz-feynman}
\usepackage{contour}

\begin{document}
\title{RPA susceptibility}

\author{Diana Csontosová}
%\email{kunes@ifp.tuwien.ac.at}
%\affiliation{TU Wien}

%\pacs{71.70.Ej,71.27.+a,75.40.Gb}
\date{\today}

\maketitle

\section{Derivation}


In order to obtain imaginary part of RPA dynamic susceptibility
$\chi(\omega)$ we use Lehmann representation of Green's function, which reads
%
\begin{equation}
    \mathfrak{G}(\mathbf{k}, iE) = \int \mathrm{d}E^{\prime}
    \frac{A(\mathbf{k}, E^{\prime})}{iE - E^{\prime}},
\end{equation}
%
and we get
%
\begin{equation}
    \begin{gathered}
    \mathrm{Im}(\chi) = \mathrm{Im}\left[ \frac{1}{\beta} \sum_{iE} \mathfrak{G}(\mathbf{k}+\mathbf{q}, iE + i\nu)\mathfrak{G}(\mathbf{k}, iE) \right]_{i\nu \rightarrow  \omega + i\delta} = \\
    \mathrm{Im}\left[ \frac{1}{\beta} \sum_{iE}  \int \int
      \mathrm{d}E^{\prime} \mathrm{d}E^{\prime \prime}
    \frac{A(\mathbf{k} + \mathbf{q}, E^{\prime})}{iE + i\nu - E^{\prime}}
    \frac{A(\mathbf{k}, E^{\prime \prime})}{iE - E^{\prime \prime}}
  \right]_{i\nu \rightarrow \omega + i\delta} = \\
  \mathrm{Im}\left[ \frac{1}{\beta} \int \int
      \mathrm{d}E^{\prime} \mathrm{d}E^{\prime \prime} A(\mathbf{k} +
      \mathbf{q}, E^{\prime}) A(\mathbf{k}, E^{\prime \prime}) \sum_{iE}
    \frac{1}{iE + i\nu - E^{\prime}}
    \frac{1}{iE - E^{\prime \prime}}
  \right]_{i\nu \rightarrow \omega + i\delta},
    \end{gathered}
  \end{equation}
%
where we sum over fermionic Matsubara frequencies and $i\nu$ is
bosonic Matsubara frequency. Now, we focus on the summation over
fermionic Matsubara frequencies of function $f(iE)$, where we substiture $iE$ for general complex
variable $z$
%
\begin{align}
    f(z) =& \frac{1}{z - z_1} \frac{1}{z - z_2} = \prod\limits_{z_j} \frac{1}{z - z_j,}
 \end{align}
%
 where
%
 \begin{align*}
    z_1 =& E^{\prime} - i\nu, \\
    z_2 =& E^{\prime \prime}.
 \end{align*}
%
For summing of function in such form we can use the summation rule
 \begin{align}
     \frac{1}{\beta} \sum_{iE} F(iE) = \sum_j n_F(z_j) \prod\limits_{j\neq j^{\prime}} \frac{1}{z_j - z_{j^{\prime}}},
 \end{align}
Note, that function $n_F(z)$ is chosen to have poles in $iE$, thus, it is
Fermi-Dirac distribution. Hence, we obtain
%
\begin{equation}
       \mathrm{Im} (\chi) =  \mathrm{Im}\left[ \frac{1}{\beta} \int \int
      \mathrm{d}E^{\prime} \mathrm{d}E^{\prime \prime} A(\mathbf{k} +
      \mathbf{q}, E^{\prime}) A(\mathbf{k}, E^{\prime \prime}) \left(
        \frac{n_F(E^{\prime} - i\nu)}{E^{\prime} - i\nu - E^{\prime
            \prime}} +  \frac{n_F(E^{\prime \prime})}{E^{\prime
            \prime} - E^{\prime} + i\nu}  \right)
  \right]_{i\nu \rightarrow \omega + i\delta},
\end{equation}
%
Due to the periodicity of Matsubara frequencies we find that
%
\begin{align*}
    n_F(E^{\prime} - i\nu) = \frac{1}{\exp{\beta (E^{\prime} - i\nu)} + 1} = \frac{1}{\exp{\beta E^{\prime}} + 1} = n_F(E^{\prime}).
\end{align*}
%
In this case the analytical continuation is rather straightforward. We only substitute $i\nu$ by $ \omega + i\delta$, where $\delta$ is infinitesimally small real number, and obtain final formula 
%
\begin{equation}
    \begin{gathered}
    \mathrm{Im} (\chi(\omega)) = \mathrm{Im}\left[ \frac{1}{\beta} \int \int
      \mathrm{d}E^{\prime} \mathrm{d}E^{\prime \prime} A(\mathbf{k} +
      \mathbf{q}, E^{\prime}) A(\mathbf{k}, E^{\prime \prime}) \left(
        \frac{n_F(E^{\prime})}{E^{\prime} - i\nu - E^{\prime
            \prime}} +  \frac{n_F(E^{\prime \prime})}{E^{\prime
            \prime} - E^{\prime} + i\nu}  \right) \right]_{i\nu \rightarrow \omega + i\delta}  = \\
    \mathrm{Im}\left[ \frac{1}{\beta} \int \int
      \mathrm{d}E^{\prime} \mathrm{d}E^{\prime \prime} A(\mathbf{k} +
      \mathbf{q}, E^{\prime}) A(\mathbf{k}, E^{\prime \prime}) \left(
        \frac{n_F(E^{\prime})}{E^{\prime} - E^{\prime \prime} - \omega - i\delta} +  \frac{n_F(E^{\prime \prime})}{E^{\prime
            \prime} - E^{\prime} + \omega +  i\delta}  \right)
    \right] = \\
     \frac{1}{\beta} \int \int
      \mathrm{d}E^{\prime} \mathrm{d}E^{\prime \prime} A(\mathbf{k} +
      \mathbf{q}, E^{\prime}) A(\mathbf{k}, E^{\prime \prime})
      \mathrm{Im} \left[
        \frac{n_F(E^{\prime}) - n_F(E^{\prime \prime})}{E^{\prime} - E^{\prime \prime} - \omega - i\delta}\right].
    \end{gathered}
\end{equation}
%
To extract imaginary part of term in bracket we use Sokhotski–Plemelj
theorem
%
\begin{equation}
\lim_{\epsilon \rightarrow 0^{+}} \frac{1}{x \pm i\epsilon} = \pm
i\pi\delta (x) + \mathcal{P}\left( \frac{1}{x} \right).
\end{equation}
%
Applying the theorem we get
%
\begin{equation}
    \begin{gathered}
 \mathrm{Im} (\chi(\mathbf{q}, \omega)) =  \frac{1}{\beta} \sum_{\mathbf{k}} \int \int
      \mathrm{d}E^{\prime} \mathrm{d}E^{\prime \prime} A(\mathbf{k} +
      \mathbf{q}, E^{\prime}) A(\mathbf{k}, E^{\prime \prime}) \left[
        n_F(E^{\prime \prime}) - n_F(E^{\prime}) \right] \pi
      \delta(E^{\prime} - E^{\prime \prime} - \omega) = \\
       \frac{1}{\beta} \sum_{\mathbf{k}} \int
      \mathrm{d}E A(\mathbf{k} +
      \mathbf{q}, E + \omega) A(\mathbf{k}, E) \left[
        n_F(E) - n_F(E + \omega) \right].
    \end{gathered}
    \label{eq:final_susc}
\end{equation}
%
In the last line of Eq.~\eqref{eq:final_susc} we recognize convolution. Note, that in order to get rid of dependence on $\mathbf{k}$-vector,
we sum over BZ. Now, we focus on integrals in Eq.~\eqref{eq:final_susc}
%
\begin{align*}
I_1 =& \int  \mathrm{d}E A(\mathbf{k} +
       \mathbf{q}, E + \omega) A(\mathbf{k}, E) n_F(E), \\
I_2 =& \int  \mathrm{d}E A(\mathbf{k} +
       \mathbf{q}, E + \omega) A(\mathbf{k}, E) n_F(E+\omega),
\end{align*}
%
where we identify convolution when we mark $H(\mathbf{k}, E) =
A(\mathbf{k}, E) n_F(E)$ in $I_1$ and $H(\mathbf{k}+\mathbf{q}, E+\omega) =
A(\mathbf{k} + \mathbf{q}, E + \omega) n_F(E+\omega)$ in $I_2$, thus
%
\begin{align*}
I_1 =& \int  \mathrm{d}E A(\mathbf{k} +
       \mathbf{q}, E + \omega) H(\mathbf{k}, E), \\
I_2 =& \int  \mathrm{d}E H(\mathbf{k} +
       \mathbf{q}, E + \omega) A(\mathbf{k}, E).
\end{align*}
%
Due to the convolution theorem, we obtain the values of integrals as
%
\begin{equation}
I = \{ A * H \}(\omega) = \mathcal{F}^{-1} \{ A^{\mathrm{FT}}
\dotproduct H^{\mathrm{FT}} \}.
\end{equation}
%
$\mathcal{F}^{-1}$ denotes inverse Fourier transform. Hence, instead of direct
calculation of convolution, we use FFT and IFFT algorithms in the
code.


The real part of susceptibility can be obtained by
Kramers-Kroning relation
%
\begin{equation}
\mathrm{Re} (\chi(\mathbf{q}, \omega)) = - \frac{1}{\pi} \mathcal{P} \int
\mathrm{d} E \frac{\mathrm{Im} (\chi(\mathbf{q}, \omega))}{E - \omega}.
\end{equation}
%

\iffalse
Using RPA approximation we can further write spectral function
$A(\mathbf{k}, E)$ as
%
\begin{equation}
A(\mathbf{k}, E) = - \frac{1}{\pi} \mathrm{Im} (G_{\mathrm{ret}}(\mathbf{k}, E)) =
\frac{1}{\pi} \frac{-\mathrm{Im}(\Sigma)}{\left( E -
    \varepsilon_{\mathbf{k}} + \mu - \mathrm{Re}(\Sigma) \right)^2 + (\mathrm{Im}(\Sigma))^2}.
\end{equation}
%
Here, $\Sigma$ marks RPA self-energy.
\fi

\end{document}
