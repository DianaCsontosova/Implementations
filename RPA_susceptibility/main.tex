\RequirePackage{luatex85}

\documentclass[preprint,prb,amsmath,superscriptaddress,showpacs]{revtex4}

%\usepackage[utf8]{inputenc}
% utf8 is not necessary once we use luatex engine

\usepackage{graphics}
\usepackage{graphicx}% Include figure files
\usepackage{dcolumn} % Align table columns on decimal point
\usepackage{bm}
\usepackage{epsfig}
\usepackage{physics}
\usepackage[linktoc=all]{hyperref}
\usepackage{natbib}
\usepackage{bibentry}
\usepackage{ bbold }
\usepackage{pgf,tikz}
\usepackage[compat=1.1.0]{tikz-feynman}
\usepackage{contour}

\begin{document}
\title{RPA susceptibility}

\author{Diana Csontosová}
%\email{kunes@ifp.tuwien.ac.at}
%\affiliation{TU Wien}

%\pacs{71.70.Ej,71.27.+a,75.40.Gb}
\date{\today}

\maketitle

\section{Derivation}

In order to obtain imaginary part of RPA dynamic susceptibility $\chi(\omega)$ we use thermal Green's function, which reads
\begin{align}
    \mathfrak{G}(\mathbf{k}, iE) = \frac{1}{iE - \varepsilon_{\mathbf{k}} - \Sigma(\mathbf{k}, iE)},
\end{align}
and we get
\begin{equation}
    \begin{gathered}
    \mathrm{Im}(\chi) = \mathrm{Im}\left[ \frac{1}{\beta} \sum_{iE} \mathfrak{G}(\mathbf{k}+\mathbf{q}, iE + i\nu)\mathfrak{G}(\mathbf{k}, iE) \right]_{i\nu \rightarrow  \omega + i\delta} = \\
    \mathrm{Im}\left[ \frac{1}{\beta} \sum_{iE} \frac{1}{iE + i\nu - \varepsilon_{\mathbf{k} + \mathbf{q}} - \Sigma(\mathbf{k} + \mathbf{q}, iE + i\nu)} \frac{1}{iE - \varepsilon_{\mathbf{k}} - \Sigma(\mathbf{k}, iE)} \right]_{i\nu \rightarrow \omega + i\delta},
    \end{gathered}
\end{equation}
where we sum over fermionic Matsubara frequencies and $i\nu$ is bosonic Matsubara frequency. Here we sum function which can be written as
\begin{align}
    F(z) =& \frac{1}{z - z_1} \frac{1}{z - z_2} = \prod\limits_{z_j} \frac{1}{z - z_j,}
 \end{align}
 where
 \begin{align*}
    z_1 =& \varepsilon_{\mathbf{k}+\mathbf{q}} + \Sigma - i\hbar \nu, \\
    z_2 =& \varepsilon_{\mathbf{k}} + \Sigma.
 \end{align*}
 Note, that $\Sigma$ is in the framework of RPA approximation constant. For summing of function in such form we can use the summation rule
 \begin{align}
     \frac{1}{\beta} \sum_{iE} F(iE) = \sum_j n_F(z_j) \prod\limits_{j\neq j^{\prime}} \frac{1}{z_j - z_{j^{\prime}}},
 \end{align}
where $n_F$ is Fermi-Dirac distribution. Hence, we obtain

\begin{align}
       \mathrm{Im} (\chi) = \mathrm{Im}\left[ \frac{1}{\beta} \left( \frac{n_F(\varepsilon_{\mathbf{k}+\mathbf{q}} + \Sigma - i\nu)}{\varepsilon_{\mathbf{k} + \mathbf{q}} + \Sigma - i \nu - \varepsilon_{\mathbf{k}} - \Sigma} + \frac{n_F(\varepsilon_{\mathbf{k}} + \Sigma)}{\varepsilon_{\mathbf{k}} + \Sigma - \varepsilon_{\mathbf{k}+\mathbf{q}} - \Sigma + i\nu}\right) \right]_{i\nu \rightarrow \omega + i\delta}.
\end{align}
According to periodicity of Matsubara frequencies we find that
\begin{align*}
    n_F(\varepsilon_{\mathbf{k}+\mathbf{q}} + \Sigma - i\nu) = \frac{1}{\exp{\beta (\varepsilon_{\mathbf{k}+\mathbf{q}} + \Sigma - i\nu)} + 1} = \frac{1}{\exp{\beta (\varepsilon_{\mathbf{k}+\mathbf{q}} + \Sigma)} + 1} = n_F(\varepsilon_{\mathbf{k}+\mathbf{q}} + \Sigma).
\end{align*}
In this case the analytical continuation is rather straightforward. We only substitute $i\nu$ by $ \omega + i\delta$, where $\delta$ is infinitesimally small real number, and obtain final formula 
\begin{equation}
    \begin{gathered}
    \mathrm{Im} (\chi(\omega)) = \mathrm{Im}\left[ \frac{1}{\beta} \left( \frac{n_F(\varepsilon_{\mathbf{k}+\mathbf{q}} + \Sigma)}{\varepsilon_{\mathbf{k} + \mathbf{q}} + \Sigma - i\nu - \varepsilon_{\mathbf{k}} - \Sigma} + \frac{n_F(\varepsilon_{\mathbf{k}} + \Sigma)}{\varepsilon_{\mathbf{k}} + \Sigma - \varepsilon_{\mathbf{k}+\mathbf{q}} - \Sigma + i\nu} \right) \right]_{i\nu \rightarrow \omega + i\delta} = \\
    \mathrm{Im}\left[ \frac{1}{\beta} \left( \frac{n_F(\varepsilon_{\mathbf{k}+\mathbf{q}} + \Sigma)}{\varepsilon_{\mathbf{k} + \mathbf{q}} - \varepsilon_{\mathbf{k}} - \omega - i\delta} + \frac{n_F(\varepsilon_{\mathbf{k}} + \Sigma)}{\varepsilon_{\mathbf{k}} - \varepsilon_{\mathbf{k}+\mathbf{q}} + \omega + i\delta} \right) \right].
    \end{gathered}
\end{equation}

\end{document}
