\RequirePackage{luatex85}

\documentclass[preprint,prb,amsmath,superscriptaddress,showpacs]{revtex4}

%\usepackage[utf8]{inputenc}
% utf8 is not necessary once we use luatex engine

\usepackage{graphics}
\usepackage{graphicx}% Include figure files
\usepackage{dcolumn} % Align table columns on decimal point
\usepackage{bm}
\usepackage{epsfig}
\usepackage{physics}
\usepackage[linktoc=all]{hyperref}
\usepackage{natbib}
\usepackage{bibentry}
\usepackage{ bbold }
\usepackage{pgf,tikz}
\usepackage[compat=1.1.0]{tikz-feynman}
\usepackage{contour}

\newcommand{\bk}{\mathbf{k}}
\newcommand{\bq}{\mathbf{q}}


\begin{document}
\title{RPA susceptibility}

\author{Diana Csontosová}
%\email{kunes@ifp.tuwien.ac.at}
%\affiliation{TU Wien}

%\pacs{71.70.Ej,71.27.+a,75.40.Gb}
\date{\today}

%\maketitle

\section{Derivation}

Here we derive the formula for bubble in paramegnetic state in
two-dimensional square lattice with one atom per unit cell,
considering one orbital per site.

In general, we can define bubble on Matsubara axis summed over
fermionic frequencies as
%
\begin{equation}
      \chi^0(\bq, i \omega_m) =
      - \frac{1}{N} \frac{1}{\beta} \sum_{\bk, i\nu} 
      \mathfrak{G}(\bk, i\nu_n)
      \mathfrak{G}(\bk + \bq, i\nu_{n} + i\omega_m).
\end{equation}
%
Using bare Green's function and standard Matsubara frequency summation
(see documentation for the derivation of RPA susceptibility) we get
%
\begin{equation}
  \begin{aligned}
  \chi^0(\bq, i \omega_m) &= - \frac{1}{N} \frac{1}{\beta} \sum_{\bk, i\nu_n} \frac{1}{i\nu_n +
    i\omega_m - E_{\bk + \bq}} \frac{1}{i\nu_n - E_{\bk}} \\
  &= - \frac{1}{N} \sum_{\bk} \frac{n_F (E_{\bk + \bq}) - n_F
    (E_{\bk})}{E_{\bk + \bq} - E_{\bk} - i\omega_m}.
  \end{aligned}
\end{equation}
In this case, we can easily perform the analytical continuation by
substituting $i\omega_m \rightarrow \omega + i\delta$ and we obtain
bubble on the real axis in the orbital basis
%
\begin{equation}
  \begin{aligned}
  \chi^0(\bq, \omega) &= - \frac{1}{N} \sum_{\bk} \frac{n_F (E_{\bk + \bq}) - n_F
    (E_{\bk})}{E_{\bk + \bq} - E_{\bk} - \omega - i\delta}.
  \end{aligned}
\end{equation}
%
Taking into account the fact, that we consider model with one atom per
unit cell and one orbital, transformation of bubble from the orbital basis to the physical
basis is very
simple (we are namely interested in component $\chi^0_{zz}$)
%
\begin{equation}
  \chi^0_{zz} = \frac{1}{2} \left(\chi^0_{\downarrow} +
    \chi^0_{\uparrow} \right).
\end{equation}
%
Since we study paramagnatic state $\chi^0_{\downarrow} = \chi^0_{\uparrow}$.
%\bibliography{library}

\end{document}
